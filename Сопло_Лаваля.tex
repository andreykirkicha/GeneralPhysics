\documentclass[12pt,a4paper]{article}
\usepackage[utf8]{inputenc}
\usepackage[T2A]{fontenc}
\usepackage[english, russian]{babel}
\usepackage{amsmath}
\usepackage{amsfonts}
\usepackage{amssymb}
\usepackage{titleps}
\usepackage{geometry}
\usepackage{hyperref}
\usepackage{float}
\usepackage{graphicx}
\usepackage{multirow}
\usepackage{hhline}

\newcommand{\w}[1]{\text{#1}}
\newcommand{\und}[1]{\underline{#1}}
\newcommand{\img}[3]{
	\begin{figure}[H]
	\begin{center}
	\includegraphics[scale=#2]{#1}
	\end{center}
	\begin{center}
 	\textit{#3}
	\end{center}
	\end{figure}
}
\newcommand{\aw}[1]{
	\begin{center}
	\textit{#1}
	\end{center}
	\n
}
\newcommand{\be}[1]{
	\begin{center}
	\boxed{#1}
	\end{center}
}
\newcommand{\beb}[1]{
	\begin{equation}
	#1
	\end{equation}
}
\newcommand{\n}{\hfill \break}
\newcommand{\x}{\cdot}

\begin{document}	
	\section*{Сопло Лаваля}
	\subsection*{Андрей Киркича, Б01-202, МФТИ, 2023}
	\n\n
	\textit{Сопло Лаваля} - канал особого профиля, который используется для разгона газа до сверхзвуковых скоростей.\n\n
	Основное применение сопла Лаваля - в ракетных двигателях.
	\subsection*{1. Уравнение Эйлера}
	Выделим участок жидкости и запишем для него второй закон Ньютона в следующем виде:
	\[\int \limits_V \vec{a} dm = \int \limits_V \vec{g} dm - \int \limits_S p d\vec{S}\]
	Знак ''$-$'' перед вторым слагаемым связан с тем, что вектор площади направлен наружу, а сила, создаваемая давлением в жидкости, считается направленной внутрь участка.\n
	Здесь учтено влияние внешнего поля (гравитации).\n\n
	Перейдя от поверхностного интеграла к объёмному по формуле Гаусса-Остроградского, можем это переписать так:
	\[\int \limits_V \vec{a} dm = \int \limits_V \vec{g} dm - \int \limits_V \vec{\nabla}p dV\]
	$\vec{\nabla}$ - оператор Набла (применяя его к давлению, получаем градиент).
	\[\int \limits_V \rho \vec{a} dV = \int \limits_V \rho \vec{g} dV - \int \limits_V \vec{\nabla}p dV\]
	\beb{\rho \vec{a} = \rho \frac{d\vec{\upsilon}}{dt} = \rho \vec{g}  - \vec{\nabla}p}\n
	В трубке мы имеем поле скоростей $\vec{\upsilon} = \vec{\upsilon}(x, y, z, t)$. Мы получили выражение $\frac{d\vec{\upsilon}}{dt}$ для некоторого участка жидкости, который как-то перемещается по трубке. Но нужно описать скорость течения в конкретной статичной точке жидкости.\n\n
	Запишем полный дифференциал скорости:
	\[d\vec{\upsilon} = \frac{\partial \vec{\upsilon}}{\partial t} dt + \frac{\partial \vec{\upsilon}}{\partial x}dx + \frac{\partial\vec{\upsilon}}{\partial y}dy + \frac{\partial \vec{\upsilon}}{\partial z}dz\]
	И разделим на $dt$:
	\[\frac{d\vec{\upsilon}}{dt} = \frac{\partial \vec{\upsilon}}{\partial t} + \frac{\partial \vec{\upsilon}}{\partial x}\frac{dx}{dt} + \frac{\partial\vec{\upsilon}}{\partial y}\frac{dy}{dt} + \frac{\partial \vec{\upsilon}}{\partial z}\frac{dz}{dt} = \frac{\partial \vec{\upsilon}}{\partial t} + \frac{\partial \vec{\upsilon}}{\partial x}\upsilon_{x} + \frac{\partial\vec{\upsilon}}{\partial y}\upsilon_{y} + \frac{\partial \vec{\upsilon}}{\partial z}\upsilon_{z}\]
	Перепишем иначе:
	\[\frac{d\vec{\upsilon}}{dt} = \frac{\partial \vec{\upsilon}}{\partial t} + (\vec{\upsilon} \x \vec{\nabla}) \x \vec{\upsilon}\]
	При этом оператор $\vec{\nabla}$ действует на вектор $\vec{\upsilon}$, стоящий справа от него.\n\n
	Выражение $(\vec{\upsilon} \x \vec{\nabla}) \x \vec{\upsilon}$ называется \textit{конвективной производной}.\n
	Выражение для $\frac{d\vec{\upsilon}}{dt}$ называется \textit{субстанциональной производной}.\n\n
	Тогда с учётом этого из (1) получаем \textit{уравнение Эйлера}:
	\be{\frac{\partial \vec{\upsilon}}{\partial t} + (\vec{\upsilon} \x \vec{\nabla}) \x \vec{\upsilon} = \vec{g} - \frac{1}{\rho} \vec{\nabla}p}
	\n
	\subsection*{2. Форма канала}
	Рассматриваем следующую модель:
	\begin{itemize}
	\item Газ идеален
	\item Поток адиабатичен
	\item Поток одномерен
	\item Массовый расход газа постоянен
	\item Влияние внешних полей принебрежимо мало
	\end{itemize}
	Тогда уравнение Эйлера приобретёт вид:
	\beb{\upsilon \frac{d\upsilon}{dx} = - \frac{1}{\rho} \frac{dp}{dx}}
	Введём \textit{число Маха} $M = \frac{\upsilon}{c}$, где $\upsilon$ - скорость газа в рассматриваемой точке, $c$ - скорость звука в газе.\n\n
	Преобразуем (2):
	\[\upsilon \frac{d\upsilon}{dx} = - \frac{1}{\rho} \frac{dp}{d\rho}\frac{d\rho}{dx}\]\n
	Из термодинамики известно, что при адиабатическом процессе $\frac{dp}{d\rho} = c^2$.
	\[\upsilon^2 \frac{d\upsilon}{dx} = - \frac{\upsilon}{\rho} c^2 \frac{d\rho}{dx}\]
	\beb{M^2 \frac{d\upsilon}{dx} = - \frac{\upsilon}{\rho} \frac{d\rho}{dx}}\n\n
	Запишем уравнение непрерывности:
	\[\rho \upsilon S = const\]
	\[\ln(\rho \upsilon S) = \ln\rho + \ln\upsilon + \ln S = const\]\n
	Продифференцируем по $x$:
	\[\frac{1}{\rho}\frac{d\rho}{dx} + \frac{1}{\upsilon}\frac{d\upsilon}{dx} + \frac{1}{S}\frac{dS}{dx} = 0\]
	\[\frac{1}{\rho}\frac{d\rho}{dx} = -\frac{1}{\upsilon}\frac{d\upsilon}{dx} - \frac{1}{S}\frac{dS}{dx}\]\n\n
	Тогда (3) принимает вид:
	\[M^2 \frac{1}{\upsilon} \frac{d\upsilon}{dx} = \frac{1}{\upsilon}\frac{d\upsilon}{dx} + \frac{1}{S}\frac{dS}{dx}\]
	\be{\frac{dS}{dx} = \frac{S}{\upsilon} \frac{d\upsilon}{dx} (M^2 - 1)}
	\newpage
	\n
	Теперь проанализируем полученную формулу. Нас интересует повышение скорости газа, поэтому $\frac{d\upsilon}{dx} > 0$.
	\begin{itemize}
	\item При $M < 1$ ($\upsilon < c$) получаем $\frac{dS}{dx} < 0$. То есть при разгоне до скорости звука сопло сужается.
	\item При $M = 1$ ($\upsilon = c$) получаем $\frac{dS}{dx} = 0$. В этой точке площадь сечения сопла минимальна. Газ здесь имеет звуковую скорость.
	\item При $M > 1$ ($\upsilon > c$) получаем $\frac{dS}{dx} > 0$.
При дальнейшем ускорении газа сопло расширяется.
	\end{itemize}
	\img{pic_1.jpg}{3}{Рисунок 1: форма сопла Лаваля}
	\subsection*{3. Характеристики и работа в среде}
	Сечение сопла, имеющее минимальную площадь, называется \textit{критическим}.\n\n
	Для ракетных двигателей вводят несколько характеристик:
	\begin{itemize}
	\item $\frac{S_{\w{вых}}}{S_{\w{кр}}}$ - \textit{степень расширения} сопла\n\n ($S_{\w{вых}}$ - площадь сечения сопла на выходе, $S_{\w{кр}}$ - площадь критического сечения)
	\item $\frac{p_{\w{вых}}}{p_{\w{атм}}}$ - \textit{степень нерасчётности} сопла\n\n ($p_{\w{вых}}$ - давление газа на выходе сопла, $p_{\w{атм}}$ - давление со стороны окружающей среды)	
	\item $I = \frac{P}{\dot{m}}$ - \textit{удельный импульс} ($\frac{\w{м}}{\w{с}}$) \n\n ($P$ - сила тяги (Н), $\dot{m}$ - массовый расход ($\frac{\w{кг}}{\w{с}}$))
	\end{itemize}
	Вообще сила тяги складывается из собственной тяги ракеты и силы со стороны внешней среды, препятствующей выходу газа. Поэтому:\n\n
	\beb{I = \frac{P_{0} + \Delta p S_{\w{вых}}}{\dot{m}} = \frac{d(m\upsilon_{0})/dt}{dm/dt} + \frac{\Delta p S_{\w{вых}}}{\dot{m}} = \upsilon_{0} \x \frac{dm/dt}{dm/dt} + \frac{\Delta p S_{\w{вых}}}{\dot{m}} = \upsilon_{0} + \frac{(p_{\w{вых}} - p_{\w{атм}}) S_{\w{вых}}}{\dot{m}}}\n\n
	$\upsilon_{0}$ - это расчётная скорость истечения газа. Видно, что в зависимости от окружающей среды итоговая скорость будет отлична от $\upsilon_{0}$.\n\n
	Будем учитывать, что давление газа по мере продвижения по соплу уменьшается.\n\n
	Выделяют несколько режимов работы в среде:
	\begin{itemize}
	\item $p_{\w{вых}} = p_{\w{атм}}$ - \textit{оптимальный (расчётный)} режим\n\n
	 В нём достигается расчётная скорость.
	\item $p_{\w{вых}} > p_{\w{атм}}$ - режим \textit{недорасширения}\n\n 
	Излишняя энергия газа расходуется на разгон внешней среды за соплом. Можно было бы использовать эту энергию для разгона самой ракеты, продлив для этого сопло.\n\n
	Недорасширение возможно в верхних слоях атмосферы, когда $p_{\w{атм}}$ мало. Для более эффективной работы двигателя создают сопловые насадки. Они продлевают сопло (тем самым увеличивая степень расширения) и больше энергии газа расходуется на пользу (то есть на разгон аппарата). Однако тут появляется другая проблема: насадки имеют массу и на разгон этой дополнительной массы тоже нужно тратить топливо. В этой ситуации учитывают конструктивные особенности всего аппарата, чтобы найти оптимальное решение.\n\n
	В вакууме недорасширение неизбежно.
	\item $p_{\w{вых}} < p_{\w{атм}}$ - режим \textit{перерасширения}\n\n 
	Газ в сопле расширяется до конца и дальше встречает слои атмосферы, которые его останавливают. По формуле (4) видно, что при этом скорость истечения газа из сопла ниже расчётной. Перерасширение становится опасным при степени нерасчётности ниже $0.4$. Возникает резкое уплотнение газа, которое может породить обратную звуковую волну. Волна будет двигаться со сверхзвуковой скоростью навстречу потоку газа и сможет достичь критического сечения. Такой процесс способен сорвать течение газа и серьёзно повредить двигатель.\n\n
	Однако конструкторы иногда сознательно идут на перерасширение в начале полёта ракеты. Как показывает практика, это оказывается более выгодно относительно трат топлива. С набором высоты атмосфера разряжается и давление $p_{\w{атм}}$ падает, наступает оптимальный режим.
	\end{itemize}
\end{document}